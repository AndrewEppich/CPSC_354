\documentclass{article}

\usepackage{amsthm}
\usepackage{amsfonts}
\usepackage{amsmath}
\usepackage{amssymb}
\usepackage{fullpage}
\usepackage{graphicx}
\usepackage[usenames]{color}
\usepackage{hyperref}
  \hypersetup{
    colorlinks = true,
    urlcolor = blue,       
    linkcolor= blue,      
    citecolor= blue,      
    filecolor= blue,     
    }
    
\usepackage{listings}

\definecolor{dkgreen}{rgb}{0,0.6,0}
\definecolor{gray}{rgb}{0.5,0.5,0.5}
\definecolor{mauve}{rgb}{0.58,0,0.82}

\lstset{frame=tb,
  language=haskell,
  aboveskip=3mm,
  belowskip=3mm,
  showstringspaces=false,
  columns=flexible,
  basicstyle={\small\ttfamily},
  numbers=none,
  numberstyle=\tiny\color{gray},
  keywordstyle=\color{blue},
  commentstyle=\color{dkgreen},
  stringstyle=\color{mauve},
  breaklines=true,
  breakatwhitespace=true,
  tabsize=3
}

\theoremstyle{plain} 
   \newtheorem{theorem}{Theorem}[section]
   \newtheorem{corollary}[theorem]{Corollary}
   \newtheorem{lemma}[theorem]{Lemma}
   \newtheorem{proposition}[theorem]{Proposition}
\theoremstyle{definition}
   \newtheorem{definition}[theorem]{Definition}
   \newtheorem{example}[theorem]{Example}
\theoremstyle{remark}    
  \newtheorem{remark}[theorem]{Remark}

\title{CPSC 354 Report}
\author{Andrew Eppich  \\ Chapman University}
\date{\today}

\begin{document}

\maketitle

\begin{abstract}

\end{abstract}

\tableofcontents

\section{Introduction}\label{intro}

\section{Homework 1}\label{homework1}

\subsection{Question 5}

\begin{lstlisting}
rw [add_zero]
rw [add_zero]
rfl
\end{lstlisting}

\subsubsection{Proof Explanation}

For this question, the Lean proof is related to the corresponding proof in mathematics because we know that we can use the additive identity property, which says that \(x + 0 = x\). By using this, we can simplify \(b + 0\) and \(c + 0\) easily to get \(a + b + c = a + b + c\), which we can determine is the same by the reflexivity property, which states that if \(a = b\), then \(a\) and \(b\) are identical. Therefore, \(a + b + c\) is identical to \(a + b + c\).

\subsection{Question 6}

\begin{lstlisting}
rw [add_zero c]
rw [add_zero b]
rfl
\end{lstlisting}


\subsection{Question 7}

\begin{lstlisting}
rw [one_eq_succ_zero]
rw [add_succ]
rw [add_zero]
rfl
\end{lstlisting}


\subsection{Question 8}

\begin{lstlisting}
rw [two_eq_succ_one]
rw [one_eq_succ_zero]
rw [add_succ]
rw[add_succ]
rw [add_zero]
rw [four_eq_succ_three]
rw [three_eq_succ_two]
rw [two_eq_succ_one]
rw [one_eq_succ_zero]
rfl
\end{lstlisting}

\subsection{Discord Question}

I was wondering if the computers use of discrete math extends to all program computations or just math computations

\section{Homework 2}\label{homework2}

\subsection{Question 1}

\begin{lstlisting}
induction n with d hd
rw [add_zero]
rfl
rw [add_succ]
rw [hd]
rfl
\end{lstlisting}

\subsection{Question 2}

\begin{lstlisting}
  induction b with d hd
  rw [add_zero]
  rw [add_zero]
  rfl
  rw [add_succ]
  rw [add_succ]
  rw [hd]
  rfl
\end{lstlisting}

\subsection{Question 3}

\begin{lstlisting}
induction b with d hd
rw [add_zero]
rw [zero_add]
rfl
rw [add_succ]
rw [hd]
rw [succ_add]
rfl
\end{lstlisting}

\subsection{Question 4}

\begin{lstlisting}
  induction a with d hd
  rw [zero_add]
  rw [zero_add]
  rfl
  rw [succ_add]
  rw [succ_add]
  rw [succ_add]
  rw [hd]
  rfl
\end{lstlisting}

\subsubsection{Explaination}

The lean proof relates to the proof in mathematics because it uses induction to solve the problem. Then the Lean proof is solved by solving the equation of the successors. Just like in mathematics it uses simple rules to change the positioning of the parenthesis so each side is exactly the same. This is exactly like how the mathematical proof would be written.

\subsection{Question 5}

\begin{lstlisting}
  induction a with d hd
  rw [zero_add]
  rw [zero_add]
  rw [add_comm]
  rfl
  rw [add_comm]
  rw [add_comm]
  rw [succ_add]
  rw [succ_add]
  rw [succ_add]
  rw [succ_add]
  rw [hd]
  rfl
\end{lstlisting}

\subsection{Discord Question}

I was wondering how discrete math and the recursive algorithms we talked about fit into a programming language and how it actually works

\end{document}
