\documentclass{article}

\usepackage{amsthm}
\usepackage{amsfonts}
\usepackage{amsmath}
\usepackage{amssymb}
\usepackage{fullpage}
\usepackage{graphicx}
\usepackage[usenames]{color}
\usepackage{hyperref}
  \hypersetup{
    colorlinks = true,
    urlcolor = blue,       
    linkcolor= blue,      
    citecolor= blue,      
    filecolor= blue,     
    }
    
\usepackage{listings}

\definecolor{dkgreen}{rgb}{0,0.6,0}
\definecolor{gray}{rgb}{0.5,0.5,0.5}
\definecolor{mauve}{rgb}{0.58,0,0.82}

\lstset{frame=tb,
  language=haskell,
  aboveskip=3mm,
  belowskip=3mm,
  showstringspaces=false,
  columns=flexible,
  basicstyle={\small\ttfamily},
  numbers=none,
  numberstyle=\tiny\color{gray},
  keywordstyle=\color{blue},
  commentstyle=\color{dkgreen},
  stringstyle=\color{mauve},
  breaklines=true,
  breakatwhitespace=true,
  tabsize=3
}

\theoremstyle{plain} 
   \newtheorem{theorem}{Theorem}[section]
   \newtheorem{corollary}[theorem]{Corollary}
   \newtheorem{lemma}[theorem]{Lemma}
   \newtheorem{proposition}[theorem]{Proposition}
\theoremstyle{definition}
   \newtheorem{definition}[theorem]{Definition}
   \newtheorem{example}[theorem]{Example}
\theoremstyle{remark}    
  \newtheorem{remark}[theorem]{Remark}

\title{CPSC 354 Report}
\author{Andrew Eppich  \\ Chapman University}
\date{\today}

\begin{document}

\maketitle

\begin{abstract}

\end{abstract}

\tableofcontents

\section{Introduction}\label{intro}

\section{Homework 1}\label{homework1}

\subsection{Question 5}

\begin{lstlisting}
rw [add_zero]
rw [add_zero]
rfl
\end{lstlisting}

\subsubsection{Proof Explanation}

For this question, the Lean proof is related to the corresponding proof in mathematics because we know that we can use the additive identity property, which says that \(x + 0 = x\). By using this, we can simplify \(b + 0\) and \(c + 0\) easily to get \(a + b + c = a + b + c\), which we can determine is the same by the reflexivity property, which states that if \(a = b\), then \(a\) and \(b\) are identical. Therefore, \(a + b + c\) is identical to \(a + b + c\).

\subsection{Question 6}

\begin{lstlisting}
rw [add_zero c]
rw [add_zero b]
rfl
\end{lstlisting}


\subsection{Question 7}

\begin{lstlisting}
rw [one_eq_succ_zero]
rw [add_succ]
rw [add_zero]
rfl
\end{lstlisting}


\subsection{Question 8}

\begin{lstlisting}
rw [two_eq_succ_one]
rw [one_eq_succ_zero]
rw [add_succ]
rw[add_succ]
rw [add_zero]
rw [four_eq_succ_three]
rw [three_eq_succ_two]
rw [two_eq_succ_one]
rw [one_eq_succ_zero]
rfl
\end{lstlisting}


\end{document}
